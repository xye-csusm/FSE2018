\section{Conclusions and Future Work}
This paper introduces using a LSTM-network for classifying a bug report as either ``predictable'' or ``unpredictable''. While a ``pedictable'' report is considered helpful to an IR-based bug locating system to find the bug, an ``unpredictable'' report is deemed unhelpful to an IR system and is discarded. Evaluation results show that our classification model helps filter out ``unpredictable'' reports and improves the ranking performance of a state-of-the-art IR system on software bug locating. Among different classification models, the LSTM-network achieves the best trade-off between precision and recall. While our approach helps for IR-based bug locating, we observe the decrease of F1-Measure and the increase of F0.5-Measure. So we conclude that if precision and recall are given equal preference, our classification model is not helpful. But in the situation that precision is more important than recall, our classification model helps while keeping a certain trade-off between recall.

In future work, we will test the effectiveness of our approach using more IR-based bug locating systems on more software projects. In parallel, We plan to explore alternative methods like \textit{sent2vec} \cite{2017arXiv170302507P} for converting bug report into vector representations. Beside, we also plan to manualy review different bug reports to get insights of what type of quality make a report to be helpful to an IR system. We plan to create features that effectively represent the quality of bug reports and combine them with neural-network-based features for more precise classiication. 