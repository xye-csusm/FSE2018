\section{Related Work}
\label{sec:related word}

Related work on word embedding in NLP was discussed in Section~\ref{sec:background}. In this section we discuss other methods for computing word similarities in software engineering and related approaches for bridging the lexical gap in software engineering tasks.

\subsection{Bug Report}

Lam et al. \cite{7372035} seeks to improve bug report handling by automating the task of associating buggy files with a bug report.  In order to overcome the lexical mismatch problem of the natural language used in bug reports not matching the terms and code tokens in source files, they combined rSVM information retrieval with deep neural networks to associate terms in bug reports to terms in source files.  Their resulting model, DnnLoc, is able to suggest likely source code files that contain the bug described in a bug report.

Huo et al. \cite{Huo:2017:EUF:3172077.3172153, Huo:2016:LUF:3060832.3060845} propose a couple of approaches to localize buggy source files from a bug report.  They first propose a novel convolutional neural network NP-CNN that leverages the structural information of source code in addition to the lexical information to accomplish this task.  They follow with another model LS-CNN that combines CNN and LSTM to additionally utilize the sequential information of source code.

Ye et al. \cite{Ye:ICSE16, Ye:FSE14} develops a learning-to-rank model to combine various features for ranking source files for bug reports.  The model is trained using source code contents, API descriptions of the code, bug-fixing history, and the code change history information of previously solved bug reports.  Further work to bridge the lexical gap between bug reports and source files was done using word embeddings to train a model to estimate semantic similarities between bug reports, source code, and API/reference documents.

Zhou et al. \cite{Zhou:2012:BFM:2337223.2337226} implemented BugLocator that locates files based on ranking by textual similarity of bug reports and source code using a revised Vector Space Model (rSVM).  Sahar et al. \cite{Saha:2013:ASE:6693093} outperforms BugLocator with BLUiR that uses structural information of code to enable more accurate bug localization.

Kim et al. \cite{Kim:2013:WFT:2554428.2554437} apply Na\"{i}ve Bayes, 

Nguyen et al. \cite{Nguyen:2011:TAN:2190078.2190181} and Lukins et al. \cite{Lukins:2010:BLU:1824820.1824850} use Latent Dirichlet Allocation (LDA), 

Rao et al. \cite{Rao:2011:RSL:1985441.1985451} apply various IR models including VSM and LDA to measure the relaitonship between bug reports and source files for recommendations.

Zhang et al. \cite{Zhang:2016:TMA:2949080.2949249} classify bug reports based on severity

Another direction for reducing effort of resolving bug reports is to automate triage of bug reports to developer(s) that are likely to resolve them. \cite{Anvik2011ReducingTE, Hu2014EffectiveBT, Zhang2013AHB, Xuan2015TowardsEB}

\subsection{Using Neural Networks to Support Software Engineering}

\cite{8255666, Huo:2017:EUF:3172077.3172153}

\subsection{Non-IR Based Bug Report Classification Techniques}

\cite{Cleve:2005:LCP:1062455.1062522, Dit:2013:IIR:2436118.2436134, Poshyvanyk:2013:CLU:2377656.2377660, Poshyvanyk:2007:FLU:1263152.1263534, Liu:2005:SSM:1081706.1081753, Jin:2013:FFL:2483760.2483763, B.Le:2016:LBF:2931037.2931049, Le:2015:IRS:2786805.2786880, Jones:2005:EET:1101908.1101949}